% Options for packages loaded elsewhere
\PassOptionsToPackage{unicode}{hyperref}
\PassOptionsToPackage{hyphens}{url}
%
\documentclass[
]{article}
\usepackage{amsmath,amssymb}
\usepackage{iftex}
\ifPDFTeX
  \usepackage[T1]{fontenc}
  \usepackage[utf8]{inputenc}
  \usepackage{textcomp} % provide euro and other symbols
\else % if luatex or xetex
  \usepackage{unicode-math} % this also loads fontspec
  \defaultfontfeatures{Scale=MatchLowercase}
  \defaultfontfeatures[\rmfamily]{Ligatures=TeX,Scale=1}
\fi
\usepackage{lmodern}
\ifPDFTeX\else
  % xetex/luatex font selection
\fi
% Use upquote if available, for straight quotes in verbatim environments
\IfFileExists{upquote.sty}{\usepackage{upquote}}{}
\IfFileExists{microtype.sty}{% use microtype if available
  \usepackage[]{microtype}
  \UseMicrotypeSet[protrusion]{basicmath} % disable protrusion for tt fonts
}{}
\makeatletter
\@ifundefined{KOMAClassName}{% if non-KOMA class
  \IfFileExists{parskip.sty}{%
    \usepackage{parskip}
  }{% else
    \setlength{\parindent}{0pt}
    \setlength{\parskip}{6pt plus 2pt minus 1pt}}
}{% if KOMA class
  \KOMAoptions{parskip=half}}
\makeatother
\usepackage{xcolor}
\usepackage[margin=1in]{geometry}
\usepackage{color}
\usepackage{fancyvrb}
\newcommand{\VerbBar}{|}
\newcommand{\VERB}{\Verb[commandchars=\\\{\}]}
\DefineVerbatimEnvironment{Highlighting}{Verbatim}{commandchars=\\\{\}}
% Add ',fontsize=\small' for more characters per line
\usepackage{framed}
\definecolor{shadecolor}{RGB}{248,248,248}
\newenvironment{Shaded}{\begin{snugshade}}{\end{snugshade}}
\newcommand{\AlertTok}[1]{\textcolor[rgb]{0.94,0.16,0.16}{#1}}
\newcommand{\AnnotationTok}[1]{\textcolor[rgb]{0.56,0.35,0.01}{\textbf{\textit{#1}}}}
\newcommand{\AttributeTok}[1]{\textcolor[rgb]{0.13,0.29,0.53}{#1}}
\newcommand{\BaseNTok}[1]{\textcolor[rgb]{0.00,0.00,0.81}{#1}}
\newcommand{\BuiltInTok}[1]{#1}
\newcommand{\CharTok}[1]{\textcolor[rgb]{0.31,0.60,0.02}{#1}}
\newcommand{\CommentTok}[1]{\textcolor[rgb]{0.56,0.35,0.01}{\textit{#1}}}
\newcommand{\CommentVarTok}[1]{\textcolor[rgb]{0.56,0.35,0.01}{\textbf{\textit{#1}}}}
\newcommand{\ConstantTok}[1]{\textcolor[rgb]{0.56,0.35,0.01}{#1}}
\newcommand{\ControlFlowTok}[1]{\textcolor[rgb]{0.13,0.29,0.53}{\textbf{#1}}}
\newcommand{\DataTypeTok}[1]{\textcolor[rgb]{0.13,0.29,0.53}{#1}}
\newcommand{\DecValTok}[1]{\textcolor[rgb]{0.00,0.00,0.81}{#1}}
\newcommand{\DocumentationTok}[1]{\textcolor[rgb]{0.56,0.35,0.01}{\textbf{\textit{#1}}}}
\newcommand{\ErrorTok}[1]{\textcolor[rgb]{0.64,0.00,0.00}{\textbf{#1}}}
\newcommand{\ExtensionTok}[1]{#1}
\newcommand{\FloatTok}[1]{\textcolor[rgb]{0.00,0.00,0.81}{#1}}
\newcommand{\FunctionTok}[1]{\textcolor[rgb]{0.13,0.29,0.53}{\textbf{#1}}}
\newcommand{\ImportTok}[1]{#1}
\newcommand{\InformationTok}[1]{\textcolor[rgb]{0.56,0.35,0.01}{\textbf{\textit{#1}}}}
\newcommand{\KeywordTok}[1]{\textcolor[rgb]{0.13,0.29,0.53}{\textbf{#1}}}
\newcommand{\NormalTok}[1]{#1}
\newcommand{\OperatorTok}[1]{\textcolor[rgb]{0.81,0.36,0.00}{\textbf{#1}}}
\newcommand{\OtherTok}[1]{\textcolor[rgb]{0.56,0.35,0.01}{#1}}
\newcommand{\PreprocessorTok}[1]{\textcolor[rgb]{0.56,0.35,0.01}{\textit{#1}}}
\newcommand{\RegionMarkerTok}[1]{#1}
\newcommand{\SpecialCharTok}[1]{\textcolor[rgb]{0.81,0.36,0.00}{\textbf{#1}}}
\newcommand{\SpecialStringTok}[1]{\textcolor[rgb]{0.31,0.60,0.02}{#1}}
\newcommand{\StringTok}[1]{\textcolor[rgb]{0.31,0.60,0.02}{#1}}
\newcommand{\VariableTok}[1]{\textcolor[rgb]{0.00,0.00,0.00}{#1}}
\newcommand{\VerbatimStringTok}[1]{\textcolor[rgb]{0.31,0.60,0.02}{#1}}
\newcommand{\WarningTok}[1]{\textcolor[rgb]{0.56,0.35,0.01}{\textbf{\textit{#1}}}}
\usepackage{graphicx}
\makeatletter
\def\maxwidth{\ifdim\Gin@nat@width>\linewidth\linewidth\else\Gin@nat@width\fi}
\def\maxheight{\ifdim\Gin@nat@height>\textheight\textheight\else\Gin@nat@height\fi}
\makeatother
% Scale images if necessary, so that they will not overflow the page
% margins by default, and it is still possible to overwrite the defaults
% using explicit options in \includegraphics[width, height, ...]{}
\setkeys{Gin}{width=\maxwidth,height=\maxheight,keepaspectratio}
% Set default figure placement to htbp
\makeatletter
\def\fps@figure{htbp}
\makeatother
\setlength{\emergencystretch}{3em} % prevent overfull lines
\providecommand{\tightlist}{%
  \setlength{\itemsep}{0pt}\setlength{\parskip}{0pt}}
\setcounter{secnumdepth}{-\maxdimen} % remove section numbering
\ifLuaTeX
  \usepackage{selnolig}  % disable illegal ligatures
\fi
\IfFileExists{bookmark.sty}{\usepackage{bookmark}}{\usepackage{hyperref}}
\IfFileExists{xurl.sty}{\usepackage{xurl}}{} % add URL line breaks if available
\urlstyle{same}
\hypersetup{
  hidelinks,
  pdfcreator={LaTeX via pandoc}}

\author{}
\date{\vspace{-2.5em}}

\begin{document}

\hypertarget{install-packages}{%
\subsection{Install Packages}\label{install-packages}}

\begin{Shaded}
\begin{Highlighting}[]
\CommentTok{\#install.packages("forecast")}
\CommentTok{\#install.packages("graphics")}
\CommentTok{\#install.packages("TTR")}
\CommentTok{\#install.packages("TSA")}
\end{Highlighting}
\end{Shaded}

\hypertarget{pemanggilan-packages}{%
\subsection{Pemanggilan Packages}\label{pemanggilan-packages}}

\begin{Shaded}
\begin{Highlighting}[]
\FunctionTok{library}\NormalTok{(}\StringTok{"forecast"}\NormalTok{)}
\end{Highlighting}
\end{Shaded}

\begin{verbatim}
## Registered S3 method overwritten by 'quantmod':
##   method            from
##   as.zoo.data.frame zoo
\end{verbatim}

\begin{Shaded}
\begin{Highlighting}[]
\FunctionTok{library}\NormalTok{(}\StringTok{"graphics"}\NormalTok{)}
\FunctionTok{library}\NormalTok{(}\StringTok{"TTR"}\NormalTok{)}
\FunctionTok{library}\NormalTok{(}\StringTok{"TSA"}\NormalTok{)}
\end{Highlighting}
\end{Shaded}

\begin{verbatim}
## Registered S3 methods overwritten by 'TSA':
##   method       from    
##   fitted.Arima forecast
##   plot.Arima   forecast
\end{verbatim}

\begin{verbatim}
## 
## Attaching package: 'TSA'
\end{verbatim}

\begin{verbatim}
## The following objects are masked from 'package:stats':
## 
##     acf, arima
\end{verbatim}

\begin{verbatim}
## The following object is masked from 'package:utils':
## 
##     tar
\end{verbatim}

\hypertarget{memanggil-data-dari-github}{%
\subsection{Memanggil data dari
github}\label{memanggil-data-dari-github}}

\begin{Shaded}
\begin{Highlighting}[]
\FunctionTok{library}\NormalTok{(rio)}
\NormalTok{data1 }\OtherTok{\textless{}{-}} \FunctionTok{import}\NormalTok{(}\StringTok{"https://raw.githubusercontent.com/Safwa40/MPDW/main/Pertemuan\%201/Data1.csv"}\NormalTok{)}
\end{Highlighting}
\end{Shaded}

\hypertarget{eksplorasi-data}{%
\subsection{Eksplorasi Data}\label{eksplorasi-data}}

\begin{Shaded}
\begin{Highlighting}[]
\FunctionTok{View}\NormalTok{(data1)}
\FunctionTok{str}\NormalTok{(data1)}
\end{Highlighting}
\end{Shaded}

\begin{verbatim}
## 'data.frame':    128 obs. of  2 variables:
##  $ Periodebulan     : int  1 2 3 4 5 6 7 8 9 10 ...
##  $ Batubara(USD/ton): num  87.5 88.3 90.1 88.6 85.3 ...
\end{verbatim}

\begin{Shaded}
\begin{Highlighting}[]
\FunctionTok{dim}\NormalTok{(data1)}
\end{Highlighting}
\end{Shaded}

\begin{verbatim}
## [1] 128   2
\end{verbatim}

\hypertarget{merubah-data-jadi-time-series}{%
\subsection{Merubah Data jadi Time
Series}\label{merubah-data-jadi-time-series}}

\begin{Shaded}
\begin{Highlighting}[]
\NormalTok{data1ts }\OtherTok{\textless{}{-}} \FunctionTok{ts}\NormalTok{(data1}\SpecialCharTok{$}\StringTok{\textquotesingle{}Batubara(USD/ton)\textquotesingle{}}\NormalTok{)}
\FunctionTok{print}\NormalTok{(data1ts)}
\end{Highlighting}
\end{Shaded}

\begin{verbatim}
## Time Series:
## Start = 1 
## End = 128 
## Frequency = 1 
##   [1]  87.55  88.35  90.09  88.56  85.33  84.87  81.69  76.70  76.89  76.61
##  [11]  78.13  80.31  81.90  80.44  77.01  74.81  73.60  73.64  72.45  70.29
##  [21]  69.69  67.26  65.70  64.65  63.84  62.92  67.76  64.48  61.08  59.59
##  [31]  59.16  59.14  58.21  57.39  54.43  53.51  53.20  50.92  51.62  52.32
##  [41]  51.20  51.81  53.00  58.37  63.93  69.07  84.89 101.69  86.23  83.32
##  [51]  81.90  82.51  83.81  75.46  78.95  83.97  92.03  93.99  94.80  94.04
##  [61]  95.54 100.69 101.86  94.75  89.53  96.61 104.65 107.83 104.81 100.89
##  [71]  97.90  92.51  92.41  91.80  90.57  88.85  81.86  81.48  71.92  72.67
##  [81]  65.79  64.80  66.27  66.30  65.93  66.89  67.08  65.77  61.11  52.98
##  [91]  52.16  50.34  49.42  51.00  55.71  59.65  75.84  87.79  84.47  86.68
## [101]  89.74 100.33 115.35 130.99 150.03 161.63 215.01 159.79 158.50 188.38
## [111] 203.69 288.40 275.64 323.91 319.00 321.59 319.22 330.97 308.20 281.48
## [121] 305.21 277.05 283.08 265.26 206.16 191.26 191.60 179.90
\end{verbatim}

\hypertarget{melihat-analisis-datanya}{%
\subsection{Melihat Analisis Datanya}\label{melihat-analisis-datanya}}

\begin{Shaded}
\begin{Highlighting}[]
\FunctionTok{summary}\NormalTok{(data1ts)}
\end{Highlighting}
\end{Shaded}

\begin{verbatim}
##    Min. 1st Qu.  Median    Mean 3rd Qu.    Max. 
##   49.42   65.78   82.20  108.26  100.74  330.97
\end{verbatim}

\begin{Shaded}
\begin{Highlighting}[]
\FunctionTok{ts.plot}\NormalTok{(data1ts, }\AttributeTok{xlab=}\StringTok{"Time Period"}\NormalTok{, }\AttributeTok{ylab=}\StringTok{"Harga Acuan Batu Bara (USD/ton)"}\NormalTok{, }
        \AttributeTok{main =} \StringTok{"Time Series Plot"}\NormalTok{)}
\FunctionTok{points}\NormalTok{(data1ts)}
\end{Highlighting}
\end{Shaded}

\includegraphics{Tugas-P1_files/figure-latex/unnamed-chunk-7-1.pdf} Pola
plot tersebut tidak berbentuk musiman ataupun konstan sehingga dicoba
pemulusan menggunakan DMA dan DES karena SMA dan SES lebih cocok untuk
data yang stasioner

\#Double Moving Average

Pembagian Data Latih da Data Uji

\begin{Shaded}
\begin{Highlighting}[]
\CommentTok{\#Pembagian data latih dan data uji dilakukan dengan perbandingan 80\% data latih dan 20\% data uji.}
\CommentTok{\#membagi data latih dan data uji}
\NormalTok{training\_ma }\OtherTok{\textless{}{-}}\NormalTok{ data1[}\DecValTok{1}\SpecialCharTok{:}\DecValTok{102}\NormalTok{,]}
\NormalTok{testing\_ma }\OtherTok{\textless{}{-}}\NormalTok{ data1[}\DecValTok{103}\SpecialCharTok{:}\DecValTok{128}\NormalTok{,]}
\NormalTok{train\_ma.ts }\OtherTok{\textless{}{-}} \FunctionTok{ts}\NormalTok{(training\_ma}\SpecialCharTok{$}\StringTok{\textasciigrave{}}\AttributeTok{Batubara(USD/ton)}\StringTok{\textasciigrave{}}\NormalTok{)}
\NormalTok{test\_ma.ts }\OtherTok{\textless{}{-}} \FunctionTok{ts}\NormalTok{(testing\_ma}\SpecialCharTok{$}\StringTok{\textasciigrave{}}\AttributeTok{Batubara(USD/ton)}\StringTok{\textasciigrave{}}\NormalTok{)}
\end{Highlighting}
\end{Shaded}

Eksplorasi data dilakukan pada keseluruhan data, data latih serta data
uji menggunakan plot data deret waktu.

\begin{Shaded}
\begin{Highlighting}[]
\CommentTok{\#eksplorasi keseluruhan data}
\FunctionTok{plot}\NormalTok{(data1ts, }\AttributeTok{col=}\StringTok{"red"}\NormalTok{,}\AttributeTok{main=}\StringTok{"Plot semua data"}\NormalTok{)}
\FunctionTok{points}\NormalTok{(data1ts)}
\end{Highlighting}
\end{Shaded}

\includegraphics{Tugas-P1_files/figure-latex/unnamed-chunk-9-1.pdf}

\begin{Shaded}
\begin{Highlighting}[]
\CommentTok{\#eksplorasi data latih}
\FunctionTok{plot}\NormalTok{(train\_ma.ts, }\AttributeTok{col=}\StringTok{"black"}\NormalTok{,}\AttributeTok{main=}\StringTok{"Plot data latih"}\NormalTok{)}
\FunctionTok{points}\NormalTok{(train\_ma.ts)}
\end{Highlighting}
\end{Shaded}

\includegraphics{Tugas-P1_files/figure-latex/unnamed-chunk-9-2.pdf}

\begin{Shaded}
\begin{Highlighting}[]
\CommentTok{\#eksplorasi data uji}
\FunctionTok{plot}\NormalTok{(test\_ma.ts, }\AttributeTok{col=}\StringTok{"blue"}\NormalTok{,}\AttributeTok{main=}\StringTok{"Plot data uji"}\NormalTok{)}
\FunctionTok{points}\NormalTok{(test\_ma.ts)}
\end{Highlighting}
\end{Shaded}

\includegraphics{Tugas-P1_files/figure-latex/unnamed-chunk-9-3.pdf}

\#Double Moving Average (DMA)\# Metode pemulusan Double Moving Average
(DMA) pada dasarnya mirip dengan SMA. Namun demikian, metode ini lebih
cocok digunakan untuk pola data trend. Proses pemulusan dengan rata rata
dalam metode ini dilakukan sebanyak 2 kali.

\begin{Shaded}
\begin{Highlighting}[]
\NormalTok{data.sma}\OtherTok{\textless{}{-}}\FunctionTok{SMA}\NormalTok{(train\_ma.ts, }\AttributeTok{n=}\DecValTok{2}\NormalTok{)}
\NormalTok{data.sma}
\end{Highlighting}
\end{Shaded}

\begin{verbatim}
## Time Series:
## Start = 1 
## End = 102 
## Frequency = 1 
##   [1]      NA  87.950  89.220  89.325  86.945  85.100  83.280  79.195  76.795
##  [10]  76.750  77.370  79.220  81.105  81.170  78.725  75.910  74.205  73.620
##  [19]  73.045  71.370  69.990  68.475  66.480  65.175  64.245  63.380  65.340
##  [28]  66.120  62.780  60.335  59.375  59.150  58.675  57.800  55.910  53.970
##  [37]  53.355  52.060  51.270  51.970  51.760  51.505  52.405  55.685  61.150
##  [46]  66.500  76.980  93.290  93.960  84.775  82.610  82.205  83.160  79.635
##  [55]  77.205  81.460  88.000  93.010  94.395  94.420  94.790  98.115 101.275
##  [64]  98.305  92.140  93.070 100.630 106.240 106.320 102.850  99.395  95.205
##  [73]  92.460  92.105  91.185  89.710  85.355  81.670  76.700  72.295  69.230
##  [82]  65.295  65.535  66.285  66.115  66.410  66.985  66.425  63.440  57.045
##  [91]  52.570  51.250  49.880  50.210  53.355  57.680  67.745  81.815  86.130
## [100]  85.575  88.210  95.035
\end{verbatim}

Dengan nilai m yang berbeda dimulai dari 4, 3, dan 2 diperoleh nilai
mape terekcil dengan nilai m = 2.

\begin{Shaded}
\begin{Highlighting}[]
\NormalTok{dma }\OtherTok{\textless{}{-}} \FunctionTok{SMA}\NormalTok{(data.sma, }\AttributeTok{n =} \DecValTok{2}\NormalTok{)}
\NormalTok{At }\OtherTok{\textless{}{-}} \DecValTok{2}\SpecialCharTok{*}\NormalTok{data.sma }\SpecialCharTok{{-}}\NormalTok{ dma}
\NormalTok{Bt }\OtherTok{\textless{}{-}} \DecValTok{2}\SpecialCharTok{/}\NormalTok{(}\DecValTok{2{-}1}\NormalTok{)}\SpecialCharTok{*}\NormalTok{(data.sma }\SpecialCharTok{{-}}\NormalTok{ dma)}
\NormalTok{data.dma}\OtherTok{\textless{}{-}}\NormalTok{ At}\SpecialCharTok{+}\NormalTok{Bt}
\NormalTok{data.ramal2}\OtherTok{\textless{}{-}} \FunctionTok{c}\NormalTok{(}\ConstantTok{NA}\NormalTok{, data.dma)}

\NormalTok{t }\OtherTok{=} \DecValTok{1}\SpecialCharTok{:}\DecValTok{26}
\NormalTok{f }\OtherTok{=} \FunctionTok{c}\NormalTok{()}

\ControlFlowTok{for}\NormalTok{ (i }\ControlFlowTok{in}\NormalTok{ t) \{}
\NormalTok{  f[i] }\OtherTok{=}\NormalTok{ At[}\FunctionTok{length}\NormalTok{(At)] }\SpecialCharTok{+}\NormalTok{ Bt[}\FunctionTok{length}\NormalTok{(Bt)]}\SpecialCharTok{*}\NormalTok{(i)}
\NormalTok{\}}
\NormalTok{data.gab2 }\OtherTok{\textless{}{-}} \FunctionTok{cbind}\NormalTok{(}\AttributeTok{aktual =} \FunctionTok{c}\NormalTok{(train\_ma.ts,}\FunctionTok{rep}\NormalTok{(}\ConstantTok{NA}\NormalTok{,}\DecValTok{26}\NormalTok{)), }\AttributeTok{pemulusan1 =} \FunctionTok{c}\NormalTok{(data.sma,}\FunctionTok{rep}\NormalTok{(}\ConstantTok{NA}\NormalTok{,}\DecValTok{26}\NormalTok{)),}\AttributeTok{pemulusan2 =} \FunctionTok{c}\NormalTok{(data.dma, }\FunctionTok{rep}\NormalTok{(}\ConstantTok{NA}\NormalTok{,}\DecValTok{26}\NormalTok{)),}\AttributeTok{At =} \FunctionTok{c}\NormalTok{(At, }\FunctionTok{rep}\NormalTok{(}\ConstantTok{NA}\NormalTok{,}\DecValTok{26}\NormalTok{)), }\AttributeTok{Bt =} \FunctionTok{c}\NormalTok{(Bt,}\FunctionTok{rep}\NormalTok{(}\ConstantTok{NA}\NormalTok{,}\DecValTok{26}\NormalTok{)),}\AttributeTok{ramalan =} \FunctionTok{c}\NormalTok{(data.ramal2, f[}\SpecialCharTok{{-}}\DecValTok{1}\NormalTok{]))}
\NormalTok{data.gab2}
\end{Highlighting}
\end{Shaded}

\begin{verbatim}
##        aktual pemulusan1 pemulusan2       At     Bt  ramalan
##   [1,]  87.55         NA         NA       NA     NA       NA
##   [2,]  88.35     87.950         NA       NA     NA       NA
##   [3,]  90.09     89.220    91.1250  89.8550  1.270       NA
##   [4,]  88.56     89.325    89.4825  89.3775  0.105  91.1250
##   [5,]  85.33     86.945    83.3750  85.7550 -2.380  89.4825
##   [6,]  84.87     85.100    82.3325  84.1775 -1.845  83.3750
##   [7,]  81.69     83.280    80.5500  82.3700 -1.820  82.3325
##   [8,]  76.70     79.195    73.0675  77.1525 -4.085  80.5500
##   [9,]  76.89     76.795    73.1950  75.5950 -2.400  73.0675
##  [10,]  76.61     76.750    76.6825  76.7275 -0.045  73.1950
##  [11,]  78.13     77.370    78.3000  77.6800  0.620  76.6825
##  [12,]  80.31     79.220    81.9950  80.1450  1.850  78.3000
##  [13,]  81.90     81.105    83.9325  82.0475  1.885  81.9950
##  [14,]  80.44     81.170    81.2675  81.2025  0.065  83.9325
##  [15,]  77.01     78.725    75.0575  77.5025 -2.445  81.2675
##  [16,]  74.81     75.910    71.6875  74.5025 -2.815  75.0575
##  [17,]  73.60     74.205    71.6475  73.3525 -1.705  71.6875
##  [18,]  73.64     73.620    72.7425  73.3275 -0.585  71.6475
##  [19,]  72.45     73.045    72.1825  72.7575 -0.575  72.7425
##  [20,]  70.29     71.370    68.8575  70.5325 -1.675  72.1825
##  [21,]  69.69     69.990    67.9200  69.3000 -1.380  68.8575
##  [22,]  67.26     68.475    66.2025  67.7175 -1.515  67.9200
##  [23,]  65.70     66.480    63.4875  65.4825 -1.995  66.2025
##  [24,]  64.65     65.175    63.2175  64.5225 -1.305  63.4875
##  [25,]  63.84     64.245    62.8500  63.7800 -0.930  63.2175
##  [26,]  62.92     63.380    62.0825  62.9475 -0.865  62.8500
##  [27,]  67.76     65.340    68.2800  66.3200  1.960  62.0825
##  [28,]  64.48     66.120    67.2900  66.5100  0.780  68.2800
##  [29,]  61.08     62.780    57.7700  61.1100 -3.340  67.2900
##  [30,]  59.59     60.335    56.6675  59.1125 -2.445  57.7700
##  [31,]  59.16     59.375    57.9350  58.8950 -0.960  56.6675
##  [32,]  59.14     59.150    58.8125  59.0375 -0.225  57.9350
##  [33,]  58.21     58.675    57.9625  58.4375 -0.475  58.8125
##  [34,]  57.39     57.800    56.4875  57.3625 -0.875  57.9625
##  [35,]  54.43     55.910    53.0750  54.9650 -1.890  56.4875
##  [36,]  53.51     53.970    51.0600  53.0000 -1.940  53.0750
##  [37,]  53.20     53.355    52.4325  53.0475 -0.615  51.0600
##  [38,]  50.92     52.060    50.1175  51.4125 -1.295  52.4325
##  [39,]  51.62     51.270    50.0850  50.8750 -0.790  50.1175
##  [40,]  52.32     51.970    53.0200  52.3200  0.700  50.0850
##  [41,]  51.20     51.760    51.4450  51.6550 -0.210  53.0200
##  [42,]  51.81     51.505    51.1225  51.3775 -0.255  51.4450
##  [43,]  53.00     52.405    53.7550  52.8550  0.900  51.1225
##  [44,]  58.37     55.685    60.6050  57.3250  3.280  53.7550
##  [45,]  63.93     61.150    69.3475  63.8825  5.465  60.6050
##  [46,]  69.07     66.500    74.5250  69.1750  5.350  69.3475
##  [47,]  84.89     76.980    92.7000  82.2200 10.480  74.5250
##  [48,] 101.69     93.290   117.7550 101.4450 16.310  92.7000
##  [49,]  86.23     93.960    94.9650  94.2950  0.670 117.7550
##  [50,]  83.32     84.775    70.9975  80.1825 -9.185  94.9650
##  [51,]  81.90     82.610    79.3625  81.5275 -2.165  70.9975
##  [52,]  82.51     82.205    81.5975  82.0025 -0.405  79.3625
##  [53,]  83.81     83.160    84.5925  83.6375  0.955  81.5975
##  [54,]  75.46     79.635    74.3475  77.8725 -3.525  84.5925
##  [55,]  78.95     77.205    73.5600  75.9900 -2.430  74.3475
##  [56,]  83.97     81.460    87.8425  83.5875  4.255  73.5600
##  [57,]  92.03     88.000    97.8100  91.2700  6.540  87.8425
##  [58,]  93.99     93.010   100.5250  95.5150  5.010  97.8100
##  [59,]  94.80     94.395    96.4725  95.0875  1.385 100.5250
##  [60,]  94.04     94.420    94.4575  94.4325  0.025  96.4725
##  [61,]  95.54     94.790    95.3450  94.9750  0.370  94.4575
##  [62,] 100.69     98.115   103.1025  99.7775  3.325  95.3450
##  [63,] 101.86    101.275   106.0150 102.8550  3.160 103.1025
##  [64,]  94.75     98.305    93.8500  96.8200 -2.970 106.0150
##  [65,]  89.53     92.140    82.8925  89.0575 -6.165  93.8500
##  [66,]  96.61     93.070    94.4650  93.5350  0.930  82.8925
##  [67,] 104.65    100.630   111.9700 104.4100  7.560  94.4650
##  [68,] 107.83    106.240   114.6550 109.0450  5.610 111.9700
##  [69,] 104.81    106.320   106.4400 106.3600  0.080 114.6550
##  [70,] 100.89    102.850    97.6450 101.1150 -3.470 106.4400
##  [71,]  97.90     99.395    94.2125  97.6675 -3.455  97.6450
##  [72,]  92.51     95.205    88.9200  93.1100 -4.190  94.2125
##  [73,]  92.41     92.460    88.3425  91.0875 -2.745  88.9200
##  [74,]  91.80     92.105    91.5725  91.9275 -0.355  88.3425
##  [75,]  90.57     91.185    89.8050  90.7250 -0.920  91.5725
##  [76,]  88.85     89.710    87.4975  88.9725 -1.475  89.8050
##  [77,]  81.86     85.355    78.8225  83.1775 -4.355  87.4975
##  [78,]  81.48     81.670    76.1425  79.8275 -3.685  78.8225
##  [79,]  71.92     76.700    69.2450  74.2150 -4.970  76.1425
##  [80,]  72.67     72.295    65.6875  70.0925 -4.405  69.2450
##  [81,]  65.79     69.230    64.6325  67.6975 -3.065  65.6875
##  [82,]  64.80     65.295    59.3925  63.3275 -3.935  64.6325
##  [83,]  66.27     65.535    65.8950  65.6550  0.240  59.3925
##  [84,]  66.30     66.285    67.4100  66.6600  0.750  65.8950
##  [85,]  65.93     66.115    65.8600  66.0300 -0.170  67.4100
##  [86,]  66.89     66.410    66.8525  66.5575  0.295  65.8600
##  [87,]  67.08     66.985    67.8475  67.2725  0.575  66.8525
##  [88,]  65.77     66.425    65.5850  66.1450 -0.560  67.8475
##  [89,]  61.11     63.440    58.9625  61.9475 -2.985  65.5850
##  [90,]  52.98     57.045    47.4525  53.8475 -6.395  58.9625
##  [91,]  52.16     52.570    45.8575  50.3325 -4.475  47.4525
##  [92,]  50.34     51.250    49.2700  50.5900 -1.320  45.8575
##  [93,]  49.42     49.880    47.8250  49.1950 -1.370  49.2700
##  [94,]  51.00     50.210    50.7050  50.3750  0.330  47.8250
##  [95,]  55.71     53.355    58.0725  54.9275  3.145  50.7050
##  [96,]  59.65     57.680    64.1675  59.8425  4.325  58.0725
##  [97,]  75.84     67.745    82.8425  72.7775 10.065  64.1675
##  [98,]  87.79     81.815   102.9200  88.8500 14.070  82.8425
##  [99,]  84.47     86.130    92.6025  88.2875  4.315 102.9200
## [100,]  86.68     85.575    84.7425  85.2975 -0.555  92.6025
## [101,]  89.74     88.210    92.1625  89.5275  2.635  84.7425
## [102,] 100.33     95.035   105.2725  98.4475  6.825  92.1625
## [103,]     NA         NA         NA       NA     NA 105.2725
## [104,]     NA         NA         NA       NA     NA 112.0975
## [105,]     NA         NA         NA       NA     NA 118.9225
## [106,]     NA         NA         NA       NA     NA 125.7475
## [107,]     NA         NA         NA       NA     NA 132.5725
## [108,]     NA         NA         NA       NA     NA 139.3975
## [109,]     NA         NA         NA       NA     NA 146.2225
## [110,]     NA         NA         NA       NA     NA 153.0475
## [111,]     NA         NA         NA       NA     NA 159.8725
## [112,]     NA         NA         NA       NA     NA 166.6975
## [113,]     NA         NA         NA       NA     NA 173.5225
## [114,]     NA         NA         NA       NA     NA 180.3475
## [115,]     NA         NA         NA       NA     NA 187.1725
## [116,]     NA         NA         NA       NA     NA 193.9975
## [117,]     NA         NA         NA       NA     NA 200.8225
## [118,]     NA         NA         NA       NA     NA 207.6475
## [119,]     NA         NA         NA       NA     NA 214.4725
## [120,]     NA         NA         NA       NA     NA 221.2975
## [121,]     NA         NA         NA       NA     NA 228.1225
## [122,]     NA         NA         NA       NA     NA 234.9475
## [123,]     NA         NA         NA       NA     NA 241.7725
## [124,]     NA         NA         NA       NA     NA 248.5975
## [125,]     NA         NA         NA       NA     NA 255.4225
## [126,]     NA         NA         NA       NA     NA 262.2475
## [127,]     NA         NA         NA       NA     NA 269.0725
## [128,]     NA         NA         NA       NA     NA 275.8975
\end{verbatim}

Hasil pemulusan menggunakan metode DMA divisualisasikan sebagai berikut

\begin{Shaded}
\begin{Highlighting}[]
\FunctionTok{ts.plot}\NormalTok{(data1ts, }\AttributeTok{xlab=}\StringTok{"Time Period "}\NormalTok{, }\AttributeTok{ylab=}\StringTok{"Batubara(USD/ton)"}\NormalTok{, }\AttributeTok{main=} \StringTok{"DMA N=2 Batubara(USD/ton)"}\NormalTok{)}
\FunctionTok{lines}\NormalTok{(data.gab2[,}\DecValTok{3}\NormalTok{],}\AttributeTok{col=}\StringTok{"green"}\NormalTok{,}\AttributeTok{lwd=}\DecValTok{2}\NormalTok{)}
\FunctionTok{lines}\NormalTok{(data.gab2[,}\DecValTok{6}\NormalTok{],}\AttributeTok{col=}\StringTok{"red"}\NormalTok{,}\AttributeTok{lwd=}\DecValTok{2}\NormalTok{)}
\FunctionTok{legend}\NormalTok{(}\StringTok{"topleft"}\NormalTok{,}\FunctionTok{c}\NormalTok{(}\StringTok{"data aktual"}\NormalTok{,}\StringTok{"data pemulusan"}\NormalTok{,}\StringTok{"data peramalan"}\NormalTok{), }\AttributeTok{lty=}\DecValTok{8}\NormalTok{, }\AttributeTok{col=}\FunctionTok{c}\NormalTok{(}\StringTok{"black"}\NormalTok{,}\StringTok{"green"}\NormalTok{,}\StringTok{"red"}\NormalTok{), }\AttributeTok{cex=}\FloatTok{0.8}\NormalTok{)}
\end{Highlighting}
\end{Shaded}

\includegraphics{Tugas-P1_files/figure-latex/unnamed-chunk-12-1.pdf}

Dilakukan pengujian data latih dan data uji dengan SSE, MSE, dan MAPE

\begin{Shaded}
\begin{Highlighting}[]
\CommentTok{\#Menghitung nilai keakuratan data latih}
\NormalTok{error\_train.dma }\OtherTok{=}\NormalTok{ train\_ma.ts}\SpecialCharTok{{-}}\NormalTok{data.ramal2[}\DecValTok{1}\SpecialCharTok{:}\FunctionTok{length}\NormalTok{(train\_ma.ts)]}
\NormalTok{SSE\_train.dma }\OtherTok{=} \FunctionTok{sum}\NormalTok{(error\_train.dma[}\DecValTok{4}\SpecialCharTok{:}\FunctionTok{length}\NormalTok{(train\_ma.ts)]}\SpecialCharTok{\^{}}\DecValTok{2}\NormalTok{)}
\NormalTok{MSE\_train.dma }\OtherTok{=} \FunctionTok{mean}\NormalTok{(error\_train.dma[}\DecValTok{4}\SpecialCharTok{:}\FunctionTok{length}\NormalTok{(train\_ma.ts)]}\SpecialCharTok{\^{}}\DecValTok{2}\NormalTok{)}
\NormalTok{MAPE\_train.dma }\OtherTok{=} \FunctionTok{mean}\NormalTok{(}\FunctionTok{abs}\NormalTok{((error\_train.dma[}\DecValTok{4}\SpecialCharTok{:}\FunctionTok{length}\NormalTok{(train\_ma.ts)]}\SpecialCharTok{/}\NormalTok{train\_ma.ts[}\DecValTok{4}\SpecialCharTok{:}\FunctionTok{length}\NormalTok{(train\_ma.ts)])}\SpecialCharTok{*}\DecValTok{100}\NormalTok{))}

\NormalTok{akurasi\_train.dma }\OtherTok{\textless{}{-}} \FunctionTok{matrix}\NormalTok{(}\FunctionTok{c}\NormalTok{(SSE\_train.dma, MSE\_train.dma, MAPE\_train.dma))}
\FunctionTok{row.names}\NormalTok{(akurasi\_train.dma)}\OtherTok{\textless{}{-}} \FunctionTok{c}\NormalTok{(}\StringTok{"SSE"}\NormalTok{, }\StringTok{"MSE"}\NormalTok{, }\StringTok{"MAPE"}\NormalTok{)}
\FunctionTok{colnames}\NormalTok{(akurasi\_train.dma) }\OtherTok{\textless{}{-}} \FunctionTok{c}\NormalTok{(}\StringTok{"Akurasi m = 2"}\NormalTok{)}
\NormalTok{akurasi\_train.dma}
\end{Highlighting}
\end{Shaded}

\begin{verbatim}
##      Akurasi m = 2
## SSE    3538.246412
## MSE      35.739863
## MAPE      5.099316
\end{verbatim}

\begin{Shaded}
\begin{Highlighting}[]
\CommentTok{\#Menghitung nilai keakuratan data uji}
\NormalTok{error\_test.dma }\OtherTok{=}\NormalTok{ test\_ma.ts}\SpecialCharTok{{-}}\NormalTok{data.gab2[}\DecValTok{103}\SpecialCharTok{:}\DecValTok{128}\NormalTok{,}\DecValTok{6}\NormalTok{]}
\NormalTok{SSE\_test.dma }\OtherTok{=} \FunctionTok{sum}\NormalTok{(error\_test.dma}\SpecialCharTok{\^{}}\DecValTok{2}\NormalTok{)}
\NormalTok{MSE\_test.dma }\OtherTok{=} \FunctionTok{mean}\NormalTok{(error\_test.dma}\SpecialCharTok{\^{}}\DecValTok{2}\NormalTok{)}
\NormalTok{MAPE\_test.dma }\OtherTok{=} \FunctionTok{mean}\NormalTok{(}\FunctionTok{abs}\NormalTok{((error\_test.dma}\SpecialCharTok{/}\NormalTok{test\_ma.ts}\SpecialCharTok{*}\DecValTok{100}\NormalTok{)))}

\NormalTok{akurasi\_test.dma }\OtherTok{\textless{}{-}} \FunctionTok{matrix}\NormalTok{(}\FunctionTok{c}\NormalTok{(SSE\_test.dma, MSE\_test.dma, MAPE\_test.dma))}
\FunctionTok{row.names}\NormalTok{(akurasi\_test.dma)}\OtherTok{\textless{}{-}} \FunctionTok{c}\NormalTok{(}\StringTok{"SSE"}\NormalTok{, }\StringTok{"MSE"}\NormalTok{, }\StringTok{"MAPE"}\NormalTok{)}
\FunctionTok{colnames}\NormalTok{(akurasi\_test.dma) }\OtherTok{\textless{}{-}} \FunctionTok{c}\NormalTok{(}\StringTok{"Akurasi m = 2"}\NormalTok{)}
\NormalTok{akurasi\_test.dma}
\end{Highlighting}
\end{Shaded}

\begin{verbatim}
##      Akurasi m = 2
## SSE   166768.37731
## MSE     6414.16836
## MAPE      27.38683
\end{verbatim}

Kedua nilai baik data latih memiliki nilai MAPE \textless10\% sehingga
data tersebut dapat dikategorikan sangat baik sedangkan data uji
memiliki nilai lebih dari 10\% pada m=2.

\#Double Exponential Smoothing (DSE)\#

Pembagian data latih dan data uji dilakukan dengan perbandingan 80\%
data latih dan 20\% data uji.

\begin{Shaded}
\begin{Highlighting}[]
\CommentTok{\#membagi training dan testing}
\NormalTok{training}\OtherTok{\textless{}{-}}\NormalTok{data1[}\DecValTok{1}\SpecialCharTok{:}\DecValTok{102}\NormalTok{,]}
\NormalTok{testing}\OtherTok{\textless{}{-}}\NormalTok{data1[}\DecValTok{103}\SpecialCharTok{:}\DecValTok{128}\NormalTok{,]}
\NormalTok{train.ts }\OtherTok{\textless{}{-}} \FunctionTok{ts}\NormalTok{(training\_ma}\SpecialCharTok{$}\StringTok{\textasciigrave{}}\AttributeTok{Batubara(USD/ton)}\StringTok{\textasciigrave{}}\NormalTok{)}
\NormalTok{test.ts }\OtherTok{\textless{}{-}} \FunctionTok{ts}\NormalTok{(testing\_ma}\SpecialCharTok{$}\StringTok{\textasciigrave{}}\AttributeTok{Batubara(USD/ton)}\StringTok{\textasciigrave{}}\NormalTok{)}
\end{Highlighting}
\end{Shaded}

Pemulusan dengan metode DES menggunakan fungsi `HoltWinters()

\begin{Shaded}
\begin{Highlighting}[]
\CommentTok{\#Lamda=0.2 dan gamma=0.2}
\NormalTok{des}\FloatTok{.1}\OtherTok{\textless{}{-}} \FunctionTok{HoltWinters}\NormalTok{(train.ts, }\AttributeTok{gamma =} \ConstantTok{FALSE}\NormalTok{, }\AttributeTok{beta =} \FloatTok{0.2}\NormalTok{, }\AttributeTok{alpha =} \FloatTok{0.2}\NormalTok{)}
\FunctionTok{plot}\NormalTok{(des}\FloatTok{.1}\NormalTok{)}
\end{Highlighting}
\end{Shaded}

\includegraphics{Tugas-P1_files/figure-latex/unnamed-chunk-16-1.pdf}

\begin{Shaded}
\begin{Highlighting}[]
\CommentTok{\#ramalan}
\NormalTok{ramalandes1}\OtherTok{\textless{}{-}} \FunctionTok{forecast}\NormalTok{(des}\FloatTok{.1}\NormalTok{, }\AttributeTok{h=}\DecValTok{10}\NormalTok{)}
\NormalTok{ramalandes1}
\end{Highlighting}
\end{Shaded}

\begin{verbatim}
##     Point Forecast     Lo 80    Hi 80     Lo 95    Hi 95
## 103       90.30620  78.13733 102.4751  71.69552 108.9169
## 104       94.72036  82.20594 107.2348  75.58120 113.8595
## 105       99.13453  86.16455 112.1045  79.29866 118.9704
## 106      103.54869  90.00676 117.0906  82.83810 124.2593
## 107      107.96286  93.72996 122.1958  86.19553 129.7302
## 108      112.37702  97.33481 127.4192  89.37194 135.3821
## 109      116.79119 100.82445 132.7579  92.37217 141.2102
## 110      121.20535 104.20374 138.2070  95.20363 147.2071
## 111      125.61952 107.47852 143.7605  97.87525 153.3638
## 112      130.03368 110.65498 149.4124 100.39651 159.6709
\end{verbatim}

\begin{Shaded}
\begin{Highlighting}[]
\CommentTok{\#Lamda=0.6 dan gamma=0.3}
\NormalTok{des}\FloatTok{.2}\OtherTok{\textless{}{-}} \FunctionTok{HoltWinters}\NormalTok{(train.ts, }\AttributeTok{gamma =} \ConstantTok{FALSE}\NormalTok{, }\AttributeTok{beta =} \FloatTok{0.1657846}\NormalTok{, }\AttributeTok{alpha =} \DecValTok{1}\NormalTok{)}
\FunctionTok{plot}\NormalTok{(des}\FloatTok{.2}\NormalTok{)}
\end{Highlighting}
\end{Shaded}

\includegraphics{Tugas-P1_files/figure-latex/unnamed-chunk-16-2.pdf}

\begin{Shaded}
\begin{Highlighting}[]
\CommentTok{\#ramalan}
\NormalTok{ramalandes2}\OtherTok{\textless{}{-}} \FunctionTok{forecast}\NormalTok{(des}\FloatTok{.2}\NormalTok{, }\AttributeTok{h=}\DecValTok{10}\NormalTok{)}
\NormalTok{ramalandes2}
\end{Highlighting}
\end{Shaded}

\begin{verbatim}
##     Point Forecast     Lo 80    Hi 80    Lo 95    Hi 95
## 103       104.5898  98.31726 110.8624 94.99677 114.1828
## 104       108.8496  99.21547 118.4837 94.11547 123.5837
## 105       113.1094 100.35880 125.8600 93.60904 132.6098
## 106       117.3692 101.53281 133.2056 93.14952 141.5889
## 107       121.6290 102.66535 140.5927 92.62659 150.6315
## 108       125.8888 103.72517 148.0525 91.99244 159.7852
## 109       130.1486 104.69720 155.6001 91.22403 169.0732
## 110       134.4084 105.57397 163.2429 90.30993 178.5069
## 111       138.6682 106.35193 170.9845 89.24472 188.0918
## 112       142.9280 107.02972 178.8264 88.02630 197.8298
\end{verbatim}

Membandingkan plot data latih dan data uji

\begin{Shaded}
\begin{Highlighting}[]
\CommentTok{\#Visually evaluate the prediction}
\FunctionTok{plot}\NormalTok{(data1ts)}
\FunctionTok{lines}\NormalTok{(des}\FloatTok{.1}\SpecialCharTok{$}\NormalTok{fitted[,}\DecValTok{1}\NormalTok{], }\AttributeTok{lty=}\DecValTok{2}\NormalTok{, }\AttributeTok{col=}\StringTok{"blue"}\NormalTok{)}
\FunctionTok{lines}\NormalTok{(ramalandes1}\SpecialCharTok{$}\NormalTok{mean, }\AttributeTok{col=}\StringTok{"red"}\NormalTok{)}
\end{Highlighting}
\end{Shaded}

\includegraphics{Tugas-P1_files/figure-latex/unnamed-chunk-17-1.pdf}

Mendapatkan nilai optimum DES

\begin{Shaded}
\begin{Highlighting}[]
\CommentTok{\#Lamda dan gamma optimum}
\NormalTok{des.opt}\OtherTok{\textless{}{-}} \FunctionTok{HoltWinters}\NormalTok{(train.ts, }\AttributeTok{gamma =} \ConstantTok{FALSE}\NormalTok{)}
\NormalTok{des.opt}
\end{Highlighting}
\end{Shaded}

\begin{verbatim}
## Holt-Winters exponential smoothing with trend and without seasonal component.
## 
## Call:
## HoltWinters(x = train.ts, gamma = FALSE)
## 
## Smoothing parameters:
##  alpha: 1
##  beta : 0.1657846
##  gamma: FALSE
## 
## Coefficients:
##         [,1]
## a 100.330000
## b   4.259804
\end{verbatim}

\begin{Shaded}
\begin{Highlighting}[]
\FunctionTok{plot}\NormalTok{(des.opt)}
\end{Highlighting}
\end{Shaded}

\includegraphics{Tugas-P1_files/figure-latex/unnamed-chunk-18-1.pdf}

\begin{Shaded}
\begin{Highlighting}[]
\CommentTok{\#ramalan}
\NormalTok{ramalandesopt}\OtherTok{\textless{}{-}} \FunctionTok{forecast}\NormalTok{(des.opt, }\AttributeTok{h=}\DecValTok{10}\NormalTok{)}
\NormalTok{ramalandesopt}
\end{Highlighting}
\end{Shaded}

\begin{verbatim}
##     Point Forecast     Lo 80    Hi 80    Lo 95    Hi 95
## 103       104.5898  98.31726 110.8624 94.99677 114.1828
## 104       108.8496  99.21547 118.4837 94.11547 123.5837
## 105       113.1094 100.35880 125.8600 93.60904 132.6098
## 106       117.3692 101.53281 133.2056 93.14952 141.5889
## 107       121.6290 102.66535 140.5927 92.62659 150.6315
## 108       125.8888 103.72517 148.0525 91.99244 159.7852
## 109       130.1486 104.69720 155.6001 91.22403 169.0732
## 110       134.4084 105.57397 163.2429 90.30993 178.5069
## 111       138.6682 106.35194 170.9845 89.24472 188.0918
## 112       142.9280 107.02972 178.8264 88.02630 197.8298
\end{verbatim}

Perhitungan akurasi pada data latih maupun data uji dengan ukuran
akurasi SSE, MSE dan MAPE.

\hypertarget{akurasi-data-latih}{%
\paragraph{Akurasi Data Latih}\label{akurasi-data-latih}}

\begin{Shaded}
\begin{Highlighting}[]
\CommentTok{\#Akurasi Data Training}
\NormalTok{ssedes.train1}\OtherTok{\textless{}{-}}\NormalTok{des}\FloatTok{.1}\SpecialCharTok{$}\NormalTok{SSE}
\NormalTok{msedes.train1}\OtherTok{\textless{}{-}}\NormalTok{ssedes.train1}\SpecialCharTok{/}\FunctionTok{length}\NormalTok{(train.ts)}
\NormalTok{sisaandes1}\OtherTok{\textless{}{-}}\NormalTok{ramalandes1}\SpecialCharTok{$}\NormalTok{residuals}
\FunctionTok{head}\NormalTok{(sisaandes1)}
\end{Highlighting}
\end{Shaded}

\begin{verbatim}
## Time Series:
## Start = 1 
## End = 6 
## Frequency = 1 
## [1]        NA        NA  0.940000 -1.615600 -5.295456 -5.257523
\end{verbatim}

\begin{Shaded}
\begin{Highlighting}[]
\NormalTok{mapedes.train1 }\OtherTok{\textless{}{-}} \FunctionTok{sum}\NormalTok{(}\FunctionTok{abs}\NormalTok{(sisaandes1[}\DecValTok{3}\SpecialCharTok{:}\FunctionTok{length}\NormalTok{(train.ts)]}\SpecialCharTok{/}\NormalTok{train.ts[}\DecValTok{3}\SpecialCharTok{:}\FunctionTok{length}\NormalTok{(train.ts)])}
                      \SpecialCharTok{*}\DecValTok{100}\NormalTok{)}\SpecialCharTok{/}\FunctionTok{length}\NormalTok{(train.ts)}

\NormalTok{akurasides}\FloatTok{.1} \OtherTok{\textless{}{-}} \FunctionTok{matrix}\NormalTok{(}\FunctionTok{c}\NormalTok{(ssedes.train1,msedes.train1,mapedes.train1))}
\FunctionTok{row.names}\NormalTok{(akurasides}\FloatTok{.1}\NormalTok{)}\OtherTok{\textless{}{-}} \FunctionTok{c}\NormalTok{(}\StringTok{"SSE"}\NormalTok{, }\StringTok{"MSE"}\NormalTok{, }\StringTok{"MAPE"}\NormalTok{)}
\FunctionTok{colnames}\NormalTok{(akurasides}\FloatTok{.1}\NormalTok{) }\OtherTok{\textless{}{-}} \FunctionTok{c}\NormalTok{(}\StringTok{"Akurasi lamda=0.2 dan gamma=0.2"}\NormalTok{)}
\NormalTok{akurasides}\FloatTok{.1}
\end{Highlighting}
\end{Shaded}

\begin{verbatim}
##      Akurasi lamda=0.2 dan gamma=0.2
## SSE                      9007.770660
## MSE                        88.311477
## MAPE                        7.925563
\end{verbatim}

\begin{Shaded}
\begin{Highlighting}[]
\NormalTok{ssedes.train2}\OtherTok{\textless{}{-}}\NormalTok{des}\FloatTok{.2}\SpecialCharTok{$}\NormalTok{SSE}
\NormalTok{msedes.train2}\OtherTok{\textless{}{-}}\NormalTok{ssedes.train2}\SpecialCharTok{/}\FunctionTok{length}\NormalTok{(train.ts)}
\NormalTok{sisaandes2}\OtherTok{\textless{}{-}}\NormalTok{ramalandes2}\SpecialCharTok{$}\NormalTok{residuals}
\FunctionTok{head}\NormalTok{(sisaandes2)}
\end{Highlighting}
\end{Shaded}

\begin{verbatim}
## Time Series:
## Start = 1 
## End = 6 
## Frequency = 1 
## [1]         NA         NA  0.9400000 -2.4858375 -3.7737239 -0.3780986
\end{verbatim}

\begin{Shaded}
\begin{Highlighting}[]
\NormalTok{mapedes.train2 }\OtherTok{\textless{}{-}} \FunctionTok{sum}\NormalTok{(}\FunctionTok{abs}\NormalTok{(sisaandes2[}\DecValTok{3}\SpecialCharTok{:}\FunctionTok{length}\NormalTok{(train.ts)]}\SpecialCharTok{/}\NormalTok{train.ts[}\DecValTok{3}\SpecialCharTok{:}\FunctionTok{length}\NormalTok{(train.ts)])}
                      \SpecialCharTok{*}\DecValTok{100}\NormalTok{)}\SpecialCharTok{/}\FunctionTok{length}\NormalTok{(train.ts)}

\NormalTok{akurasides}\FloatTok{.2} \OtherTok{\textless{}{-}} \FunctionTok{matrix}\NormalTok{(}\FunctionTok{c}\NormalTok{(ssedes.train2,msedes.train2,mapedes.train2))}
\FunctionTok{row.names}\NormalTok{(akurasides}\FloatTok{.2}\NormalTok{)}\OtherTok{\textless{}{-}} \FunctionTok{c}\NormalTok{(}\StringTok{"SSE"}\NormalTok{, }\StringTok{"MSE"}\NormalTok{, }\StringTok{"MAPE"}\NormalTok{)}
\FunctionTok{colnames}\NormalTok{(akurasides}\FloatTok{.2}\NormalTok{) }\OtherTok{\textless{}{-}} \FunctionTok{c}\NormalTok{(}\StringTok{"Akurasi lamda=1 dan gamma=0.1657846"}\NormalTok{)}
\NormalTok{akurasides}\FloatTok{.2}
\end{Highlighting}
\end{Shaded}

\begin{verbatim}
##      Akurasi lamda=1 dan gamma=0.1657846
## SSE                          2376.006342
## MSE                            23.294180
## MAPE                            4.184672
\end{verbatim}

Hasil akurasi dari data latih didapatkan skenario 2 dengan lamda=1 dan
gamma=0.1657846 memiliki hasil yang lebih baik. Namun untuk kedua
skenario dapat dikategorikan peramalan sangat baik berdasarkan nilai
MAPE-nya.

\end{document}
